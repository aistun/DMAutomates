\documentclass[a4paper, 11pt]{article}

\usepackage[utf8]{inputenc}
\usepackage[T1]{fontenc}
\usepackage[french]{babel}
\usepackage[margin=2.5cm]{geometry}

\usepackage{amsmath}
\usepackage{amssymb}
\usepackage[french, onelanguage]{algorithm2e}
\usepackage{graphicx}
\usepackage{listings}

\title{Automates et Applications : Rapport}
\author{Josué Moreau}
\date{8 Mars 2020}

\SetKwProg{Fn}{fonction}{}{fin}
\SetKwBlock{Init}{initialement}{fin}
\SetKwFor{ForAll}{pour tout}{faire}{fin}
\SetKwBlock{Pre}{précondition}{fin}
\SetKwBlock{Post}{postcondition}{fin}
\SetKwFor{ForEach}{pour chaque}{faire}{fin}

\newcommand\et{\text{ et }}
\newcommand\ou{\text{ ou }}

\begin{document}

%% Configuration pour les pseudo codes d'algorithmes
% \RestyleAlgo{ruled}
\SetAlgoVlined
% \LinesNumbered
\DontPrintSemicolon

\maketitle

\section*{2. Validation top-down non déterministe}

\begin{enumerate}
\item Un run d'un automate d'arbre $A = (Q, \Sigma, \delta, I, F)$ pour un arbre
  $t \in \mathcal{T}(\Sigma)$ est une fonction $r : dom(t) \to Q$ telle que
  $\forall p \in dom(t), (r(p1), ..., r(pk)) \in \delta(L, r(p))$ et
  $t(p) \in L$, avec $|t(p)| = k$. Un run est dit acceptant si et seulement si
  $r(\varepsilon) \in I$ et $\forall p \in dom(t), |t(p)| = 0 \to r(p) \in F$.
\item \hfill\\
  \begin{algorithm}[H]
    \Fn{$\texttt{validate\_td}(A, t, p, q)$}{
      \eIf{$t(p) = \#$}{
        \Return $(t(p) \in A.F)$\;
      }{
        \ForEach{transition $(L, q', q_1, q_2)$ dans $A$}{
          \If{$q = q' \et t(p) \in L$}{
            \Return $\texttt{validate\_td}(A, t, \texttt{first-child}(p), q_1) \et$\\
            \quad $\texttt{validate\_td}(A, t, \texttt{next-sibling}(p), q_2)$\;
          }
        }
        \Return faux\;
      }
    }
  \end{algorithm}
\item
  \begin{itemize}
  \item La complexité de tester l'existence d'un $q$ dans $I$ qui
    satisfait une certaine propriété se fait en temps $O(|I|)$.
  \item Si l'on appelle $E$ le nombre d'étiquettes définies explicitement dans
    le fichier de types de l'automate $A$ i.e. le nombre d'étiquettes dans les
    ensembles $L$ finis et dans les complémentaires des ensembles $L$ co-finis.
  \item La complexité de \texttt{validate\_td}$(A, t, \varepsilon, q)$ est en\\
    $O(|A| \times (\log E \text{ } + \text{ complexité des appels récursifs}))$.
    Ainsi, la complexité de l'expression
    \texttt{validate\_td}$(A, t, \varepsilon, q)$ est
    $\underbrace{|A| \times ... \times |A|}_{\max_{p \in dom(t)} |p|}$ où $|p|$
    est la longueur du chemin $p$. Le pire des cas est celui où l'arbre a une
    forme filaire, dans ce cas $\max_{p \in dom(t)} |p| = |t|$ (si l'on ne
    compte les noeuds vides).
  \item Donc, la complexité de l'expression
    $\exists q \in I, \texttt{validate\_td}(A, t, p, q)$ est en
    $O(|I||A|^{|t|})$.
  \end{itemize}
\end{enumerate}

\section*{3. Validation bottom-up}

\begin{enumerate}
\item \hfill\\
  \begingroup
  \LinesNumbered
  \begin{algorithm}[H]
    \Fn{$\texttt{validate\_bu}(A, t, p)$}{
      \eIf{$t(p) = \#$}{
        $R \leftarrow \emptyset$\;
        \ForEach{transition $(L, q, \_, \_)$ dans $A$}{
          \If{$\# \in L$}{
            ajouter $q$ à $R$\;
          }
        }
        \Return $R$\;
      }{
        $R_1 \leftarrow \texttt{validate\_bu}(A, t, \texttt{first-child}(p))$\;
        $R_2 \leftarrow \texttt{validate\_bu}(A, t, \texttt{next-sibling}(p))$\;
        $R \leftarrow \emptyset$\;
        \ForEach{transition $(L, q, q_1, q_2)$ dans $A$}{
          \If{$q_1 \in R_1 \et q_2 \in R_2 \et t(p) \in L$}{
            ajouter $q$ à $R$\;
          }
        }
        \Return $R$\;
      }
    }
  \end{algorithm}
  \endgroup
\item
  \begin{itemize}
  \item Avec les appels récursifs, les calculs des lignes 11 à 15 sont exécutés
    exactement une fois pour chaque noeud de l'arbre $t$, donc exactement $|t|$
    fois (on ne compte pas dans $|t|$ les noeuds vides).
  \item À chaque fois, les calculs des lignes 11 à 15 se font en temps
    $O(\log E \times (|R_1| + |R_2|))$.
  \item Or, $|R_1| = O(|A|)$ et $|R_2| = O(|A|)$ car ils sont construits de
    telle manière que pour chaque transition de $A$ on ajoute au plus un élément
    dans $R_1$ (idem pour $R_2$).
  \item Donc, la complexité totale de l'expression
    $\texttt{validate\_bu}(A, t, \varepsilon)$ est en
    $O(|t| \times |A| \times \log E)$.
  \item Le calcul de l'intersection
    $\texttt{validate\_bu}(A, t, \varepsilon) \cap I$ se fait en temps\\
    $O(|\texttt{validate\_bu}(A, t, \varepsilon)| \times I) = O(|A|)$ (en
    supposant qu'il existe un ordre sur les états).
  \item Ainsi, la complexité de l'expression
    $\texttt{validate\_bu}(A, t, \varepsilon) \cap I$ est en
    $O(|t| \times |A| \times \log E)$.
  \end{itemize}
\item \hfill\\
  \begingroup
  \LinesNumbered
  \begin{algorithm}[H]
    \Fn{$\texttt{validate\_opt}(A, t, p, P)$}{
      \uIf{$P = \emptyset$}{
        \Return $\emptyset$\;
      }
      \uElseIf{$t(p) = \#$}{
        $R \leftarrow \emptyset$\;
        \ForEach{transition $(L, q, \_, \_)$ dans $A$}{
          \If{$\# \in L$}{
            ajouter $q$ à $R$\;
          }
        }
        \Return $R$\;
      }
      \Else{
        $P' \leftarrow \emptyset$\;
        $P_1 \leftarrow \emptyset$\;
        $P_2 \leftarrow \emptyset$\;
        \ForEach{transition $(L, q, q_1, q_2)$ dans $A$}{
          \If{$q \in P \et t(p) \in L$}{
            ajouter $q$ à $P'$\;
            ajouter $q_1$ à $P_1$\;
            ajouter $q_2$ à $P_2$\;
          }
        }
        $R_1 \leftarrow \texttt{validate\_opt}(A, t, \texttt{first-child}(p), P_1)$\;
        \If{$R_1 = \emptyset$}{
          \Return $\emptyset$\;
        }
        $R_2 \leftarrow \texttt{validate\_opt}(A, t, \texttt{next-sibling}(p), P_2)$\;
        \If{$R_2 = \emptyset$}{
          \Return $\emptyset$\;
        }

        $R \leftarrow \emptyset$\;
        \ForEach{transition $(L, q, q_1, q_2)$ dans $A$}{
          \If{$q \in P' \et q_1 \in R_1 \et q_2 \in R_2$}{
            ajouter $q$ à $R$\;
          }
        }
        \Return $R$\;
      }
    }
  \end{algorithm}
  \endgroup

  Si $P$ est vide, alors la fonction doit immédiatement rendre $\emptyset$. Si
  $R$ est vide, dans le cas où il provient d'un appel récursif (il s'agit donc
  d'un $R_1$ ou $R_2$) alors nécessairement $|R| = 0$, donc on fait remonter ce
  résultat sans faire aucun calculs.
\end{enumerate}

\section*{4. Compilation}

\subsection*{Compilation des étiquettes}

\begin{algorithm}[H]
  \Fn{$\texttt{compile\_label}(l)$}{
    \uIf{$l$ est un mot $m$}{
      \Return $\{m\}$\;
    }
    \uElseIf{$l$ est $*$}{
      \Return $\Sigma \setminus \emptyset$\;
    }
    \Else{
      // $l$ est de la forme $\sim l_1 \sim ... \sim l_n$\;
      \Return $\Sigma \setminus \{l_1, ..., l_n\}$\;
    }
  }
\end{algorithm}

\subsection*{Arbres XML $\rightarrow$ Arbres binaires}

Il est possible de représenter des arbres XML avec des arbres binaires.
En effet, considérons le fichier XML suivant :
\\\texttt{<L1>}
\\\null\quad \texttt{<F1></F1>}
\\\null\quad \texttt{<L2>}
\\\null\qquad \texttt{<F2></F2>}
\\\null\qquad \texttt{<L3>}
\\\null\qquad\quad \texttt{<F3></F3>}
\\\null\qquad\quad \texttt{<F4></F4>}
\\\null\qquad \texttt{</L3>}
\\\null\qquad \texttt{<F5></F5>}
\\\null\quad \texttt{</L2>}
\\\texttt{</L1>}

L'arbre correspondant à ce fichier XML est le suivant :\\
\begin{center}
  \includegraphics{figure-1}
\end{center}

On peut alors rajouter des $\#$ et les liens rouges, qui lient un noeud et son
premier fils, ainsi qu'un noeud et son premier frère de droite. On obtient
alors :\\
\begin{center}
  \includegraphics{figure-2}
\end{center}

En extrayant enfin l'arbre rouge, on obtient l'arbre binaire suivant :\\
\begin{center}
  \includegraphics{figure-3}
\end{center}

\subsection*{Compilation des types}

Pour compiler les types, on a déjà défini la fonction \texttt{compile\_label}
qui compile une étiquette d'une définition de type. Il faut maintenant compiler
l'expression régulière de la définition de type. Pour cela, on va considérer que
l'expression régulière est une expression régulière telle qu'on les a considéré
pour les langages rationnels mais avec comme ensemble de symboles les
identifiants des types. Il est alors possible de compiler cette expression
régulière en un automate fini non-déterministe $\mathcal{A}_r$ avec la
construction de Glushkov. On pourra ensuite se servir de cet automate
$\mathcal{A}_r$ pour construire les transitions correspondant à la définition de
type dont on a construit l'automate pour la regexp.

On défini $q_\#$ un état spécial qui ne doit se trouver que sur un feuille vide
$\#$. On va considérer que tout noeud vide peut passer à l'état $q_\#$ et que
c'est le seul état dans lequel un noeud vide peut passer. Cette simplification
n'a aucune incidence sur la reconnaissance d'arbre par l'automate.

Tout d'abord, pour chaque définition de types \texttt{type} $a$ \texttt{=} $l_a$
\texttt{[ ... ]} on compile son étiquette et on crée un état associé à cette
définition. Cet état sera égal à $q_\#$ lorsque l'expression régulière de la
définition de type est vide, sinon il sera différent de tous les autres états. À
noter que comme il peut y avoir plusieurs définitions de types pour un même
type, un même type correspond à autant d'étiquettes compilées et d'états.

L'idée derrière ces états associés aux définitions de types est qu'ils vont
servir représenter le fait qu'un sous-arbre est reconnue par l'expression
régulière de la définition de type associé.

Ensuite pour chaque définition de type, on construit l'automate fini
$\mathcal{A}_r$ correspondant à l'expression régulière $r$ associée à la
définition de type. Cet automate aura pour unique état initial l'état associé à
la définition du type.

Puis, pour chaque transition $q \xrightarrow{a} q'$ dans $\mathcal{A}_r$ où $a$
est un identifiant de type, on va ajouter la transition $(L_a, q, q_a, q')$ pour
tout $L_a$ et $q_a$ associées au type $a$. On ajoute également la transition
$(L_A, q, q_a, q_\#)$ pour tout $L_a$ et $q_a$ associés au type $a$ si $q'$ est
un état final dans l'automate $\mathcal{A}_r$.

L'ensemble des états est ceux créés précédemment pour chaque définition de types
ainsi que tous les états créés lors des automates finis associés aux expressions
régulières. On rajoute enfin un dernier état, $accept$ et on rajoute les
transitions $(L_a, accept, q_a, q_\#)$ à pour $L_a$ et $q_a$ associés au type de
la racine.

Avec cette construction, plusieurs états peuvent être ajoutés ainsi dans
l'automate d'arbre mais sans aucune possibilité de les avoir un jour sur un
noeud. On pourra alors les supprimer, ainsi que toutes les transitions dans
lesquels ils figurent.

On a donc le pseudo-code suivant :

\begin{algorithm}[H]
  \Fn{$\texttt{compile\_types}(\texttt{types\_defs}, t_{\text{initial}})$}{
    $Q \leftarrow \emptyset$\;
    $T \leftarrow \emptyset$\;
    \ForEach{définition de type $tdef$ : \texttt{type} $t$ \texttt{=} $l$ \texttt{[} $r$ \texttt{]}}{
      créer un nouvel état $q_{tdef}$\;
      \If{$r = \emptyset$}{
        $q_{tdef} \leftarrow q_\#$\;
      }
      $L_{tdef} \leftarrow \texttt{compile\_label}(l)$\;
    }
    \ForEach{définition de type $tdef$ : \texttt{type} $t$ \texttt{=} $l$ \texttt{[} $r$ \texttt{]}}{
      $(Q_{tdef}, F_{tdef}, T_{tdef}) \leftarrow$ NFA (par la construction de Glushkov) pour l'expression
      régulière $r$ avec $q_{tdef}$ comme unique état initial\;
      $Q \leftarrow Q \cup Q_{tdef}$\;
      \ForEach{transition $q \xrightarrow{a} q'$ dans $T_{tdef}$}{
        \ForEach{$(q_a, L_a)$ associés au type $a$}{
          ajouter $(L_{a}, q, q_{a}, q')$ à $T$\;
          \If{$q' \in F_{tdef}$}{
            ajouter $(L_{a}, q, q_{a}, q_\#)$ à $T$\;
          }
        }
      }
      \If{$t = t_{\text{initial}}$}{
        ajouter $(l_{tdef}, accept, q_{tdef}, q_\#)$ à $T$\;
      }
    }

    \Return $(Q, T, {accept})$\;
  }
\end{algorithm}

\subsection*{Taille de l'automate d'arbre}

\begin{itemize}
\item Soit une étiquette $l$. Si $l$ est un mot $m$, alors on défini sa taille
  comme égale est 1, si $l = *$, alors on défini sa taille comme égale à 1, si
  $l = \sim l_1 \sim ... \sim l_n$, alors on défini la taille de $l$ comme égale
  à $n$. On remarque que cette taille est aussi celle associée à une étiquette
  compilée au vu de l'algorithme \texttt{compile\_label}. Cette taille est
  inférieure ou égale à $E$ où $E$ est le nombre d'étiquettes définies
  explicitement dans le fichier de type.
\item On défini ensuite inductivement la taille d'une expression régulière $r$ :
  \begin{itemize}
  \item Si $r = t$ où $t$ est un type, alors $|r| = 1$
  \item Si $r = r_1^*$, alors $|r| = 1 + |r_1|$
  \item Si $r = r_1?$, alors $|r| = 1 + |r_1|$
  \item Si $r = r_1r_2$, alors $|r| = |r_1| + |r_2|$
  \item Si $r = r_1 | r_2$, alors $|r| = |r_1| + |r_2|$
  \end{itemize}
\item On peut ensuite définir la taille d'une définition de type $tdef$ :
  \texttt{type} $t$ \texttt{=} $l$ \texttt{[} $r$ \texttt{]}.
  $|tdef| = |l| + |r|$.
\item De plus, la construction de Glushkov construit un état pour chaque facteur
  de longueur 2, et qu'il y a au plus $K^2$ facteurs de longueur 2, où $K$ est
  le nombre de types (sans compter deux fois un même type) définis dans le
  fichier de types. Si on appelle maintenant $N$ le nombre de définitions de
  types dans le fichier (cette fois en comptant donc autant de fois un type
  qu'il y a de définitions associées à lui), on déduit alors qu'on ajoute au
  plus $O(NK^2)$ transitions à l'automate d'arbre que l'on est en train de
  construire. En effet, pour chaque transition de $\mathcal{A}_r$, il faut
  ajouter autant de transitions dans l'automate d'arbre qu'il y a de couples
  $(q_a, L_a)$, tels que définis précédemment, associés au type $a$, qui est
  l'étiquette de la transition de $\mathcal{A}_r$.
\item On ajoute également transitions de $accept$ vers les états associés au
  type initial, il y en a au plus $N$ où $N$ est le nombre de définitions de
  types.
\item On a donc un automate de taille
  $O(E \times (N + N^2 \times K^2)) = O(EN^2K^2)$ car $K \geq 1$. La taille de
  l'automate est donc $O(EN^4)$ car $K \leq N$ (le nombre de types définis (sans
  compter deux fois un même type) est inférieur ou égal au nombre de définitions
  de types).
\end{itemize}


\end{document}
% LocalWords:  identifiants inductivement
